\chapter{Posts}

A post is a piece of content that is created and published on a regular basis. They are similar to pages, but are better suited to publishing information sorted by date order, such as blog posts, announcements and press releases.

\section{Making a post}

Posts are made by creating a Markdown file in the posts directory:

\begin{itemize}
  \item \texttt{website/\_posts/2018-01-01-a-post.markdown}
  \item \texttt{website/\_posts/2018-07-01-another-post.markdown}
  \item \texttt{website/\_posts/2019-01-01-yet-another-post.markdown}
\end{itemize}

\section{Post frontmatter}

Like pages, all posts require frontmatter at the start of the file to determine their layout and title, as well as their date. The post frontmatter can also include the author, categories, and an excerpt. This is in YAML format and will typically look like this:

\begin{quote}
  \texttt{%
    ---\\
    layout: post\\
    title: "Post Title"\\
    date: YYYY-MM-DD\\
    author: "Post Author"\\
    categories:\\
    - category1\\
    - category2\\
    excerpt: "Sed posuere consectetur est at lobortis. Cum sociis natoque penatibus et magnis dis parturient montes, nascetur ridiculus mus. Lorem ipsum dolor sit amet, consectetur adipiscing elit."\\
    ---
  }
\end{quote}

\section{Blog posts, announcements and press releases}

All posts are published at \texttt{rzzt.io/blog/} regardless of category, but adding the category `blog' helps to identify blog posts. Posts with the category `announcement' are published at \texttt{rzzt.io/announcements/} and posts with the category `media' (for press releases) are published at \texttt{rzzt.io/media/}. The relevant category (`blog', `announcement' or `media') should be placed first on a post's list of categories. Other categories may be added as appropriate.
